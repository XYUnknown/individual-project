    
\documentclass[11pt]{article}
\usepackage{times}
\usepackage{fullpage}
\usepackage{url}

\usepackage{amssymb}
\usepackage{newunicodechar}
\usepackage{bbm}
\usepackage[greek, english]{babel}
\usepackage{textgreek}

% This handles the translation of unicode to latex:

\usepackage{ucs}
\usepackage[utf8x]{inputenc}
\usepackage{autofe}

% Some characters that are not automatically defined
% (you figure out by the latex compilation errors you get),
% and you need to define:

%\DeclareUnicodeCharacter{8988}{\ensuremath{\ulcorner}}
%\DeclareUnicodeCharacter{8989}{\ensuremath{\urcorner}}
%\DeclareUnicodeCharacter{8803}{\ensuremath{\overline{\equiv}}}
%\DeclareUnicodeCharacter{03BB}{\ensuremath{\lambda}}

% Add more as you need them (shouldn't happen often).

% Using "\newenvironment" to redefine verbatim to
% be called "code" doesn't always work properly. 
% You can more reliably use:

\usepackage{fancyvrb}

\DefineVerbatimEnvironment
    {code}{Verbatim}
    {} % Add fancy options here if you like.
  
\title{ Proving the Correctness of Rewrite Rules in LIFT Rewite-Based System }
\author{ Xueying Qin - 2335466Q }

\begin{document}
\maketitle

\section{Status report}

\subsection{Proposal}\label{proposal}

\subsubsection{Motivation}\label{motivation}
The rewrite system of LIFT systematically transforms high-level algorithmic patterns into low-level 
high performance OpenCL code with equivalent functionality. During this process, a set of rewrite 
rules are applied. Ensuring the correctness of these rules is important for ensuring the functionality 
is not altered during the rewrite process. Currently, the correctness of these rules is only proven 
correct on paper and only for a subset of the used rules. Thus, in this project I will develop 
mechanical proofs in Agda to show the correctness of the rewrite rules.

\subsubsection{Aims}\label{aims}
First of all, I would like to design effective semantics for primitive function to simplify the process of 
developing proofs. Secondly, developing effective lemmas to keep the modularity of proofs is also essential 
for this project. Thirdly, it is important to verify if the rules are correct and potentially propose 
strategies to fix issues if they occurs.

\subsection{Progress}\label{progress}
A set of primitives are defined and basic rewrite rules are proven. Details are recorded in the 
project repository: \small{\url{https://github.com/XYUnknown/individual-project/blob/master/list.lagda.md}}
\begin{itemize}
    \item Primitives
\begin{code}
id : {T : Set} → T → T
map : {n : ℕ} → {s : Set} → {t : Set} → (s → t) → Vec s n → Vec t n
take : (n : ℕ) → {m : ℕ} → {t : Set} → Vec t (n + m) → Vec t n
drop : (n : ℕ) → {m : ℕ} → {t : Set} → Vec t (n + m) → Vec t m
split : (n : ℕ) → {m : ℕ} → {t : Set} → Vec t (n * m) → 
        Vec (Vec t n) m
join : {n m : ℕ} → {t : Set} → Vec (Vec t n) m → Vec t (n * m)
slide : {n : ℕ} → (sz : ℕ) → (sp : ℕ)→ {t : Set} → 
        Vec t (sz + n * (suc sp)) → Vec (Vec t sz) (suc n)
reduceSeq : {n : ℕ} → {s t : Set} → (s → t → t) → t → Vec s n → t
reduce : {n : ℕ} → {t : Set} → (M : CommAssocMonoid t) → Vec t n → t
partRed : (n : ℕ) → {m : ℕ} → {t : Set} → (M : CommAssocMonoid t) → 
          Vec t (suc m * n) → Vec t (suc m)
\end{code}
    \item Rewite rules
    \begin{itemize}
        \item Identity rules
        \item Fusion rules
        \item Simplification rules
        \item Split-join rule
        \item Reduction rule
        \item Partial reduction rules
    \end{itemize}
\end{itemize}

\subsection{Problems and risks}\label{problems-and-risks}

\subsubsection{Problems}\label{problems}
The dependent types and pattern matching in Agda overcomplicated defining primitives and 
developing proofs. For example, when reasoning about the size of a vector, it has been tedious to 
prove something like the size $m * suc \;n$ and $suc \;n * m$ are the same. We made use of the 
built-in \texttt{REWRITE} feature to override some pattern matching, which helped simplify 
the semantics and proofs.
\\[0.2cm]
The definitions of \texttt{reduce} and \texttt{partRed} require an arbitrary associtive and commutative 
operator with an identity element. It was difficult to solely define such an operator with the identity element in Agda. 
We resolved this issue by declaring a typeclass in Agda to enmulate this operator with certain properties. 
This ensured the modularity of proofs.

\subsubsection{Risks}\label{risks}
The built-in \texttt{REWRITE} is an unsafe feature in Agda, misusing it can potentially break the soundness of 
the type system in Agda. The mitigation of this risk is we must justify, i.e., provide proofs for the pattern 
matching to be rewritten into the system whenever we need to use this feature, to ensure the soundness of Agda type system.

\subsection{Plan}\label{plan}
In the first semester, I have finished the basic features proposed in the project proposal. In the first half of 
next semester, firstly, I will develop the semantics for more complicated primitive such as \texttt{iterate}, 
\texttt{reorder} and \texttt{transpose}. Sencondly, relevant rewrite rules will be proven. Moreover, we might 
look into the correctness of ELEVATE strategies. In the second half of semester, I will write the finial 
dissertation.

    
\subsection{Ethics and data}\label{ethics}
This project does not involve human subjects or data. No approval required.


\end{document}
